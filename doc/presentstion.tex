%\usecolortheme{seagull}
\documentclass[10pt, xcolor=x11names]{beamer}
\useoutertheme{infolines}
\usefonttheme[onlymath]{serif}
\setbeamertemplate{headline}[default]
\setbeamertemplate{navigation symbols}{}
\mode<beamer>{\setbeamertemplate{blocks}[rounded][shadow=true]}
\setbeamercovered{transparent}
%\setbeamercolor{block body example}{fg=blue, bg=black!20}

\usepackage[utf8]{inputenc}
\usepackage[]{csquotes}
\usepackage{amsmath}
\usepackage{outlines}
\usepackage[backend=biber]{biblatex}
\addbibresource{thesis.bib}

% Removes icon in bibliography
\setbeamertemplate{bibliography item}{}
\nocite{*}

%\usepackage{tikz, wasysym}
%\usepackage{graphicx}
%\usetikzlibrary{automata,positioning}
%\usepackage{hyperref}
%\usepackage{amsfonts}
%\usepackage{csquotes}
%\usepackage{tikz}
%\usetikzlibrary{arrows}
%\usetikzlibrary{arrows.meta}
%\usetikzlibrary{positioning}
%\usepackage{wrapfig}
%\usepackage{pgfplots}

%\usepackage{mathtools}
%\usepackage{xcolor}
%\usepackage{amsfonts}
%\usepackage{amssymb}
%\usepackage{makeidx}
%\usepackage{graphicx}


\usepackage{hyperref}
\author{Sven Fiergolla \& Tobias Dahlem}
\title[]{btrfs \& F2FS vs. ext4}
\subtitle[short version]{}
\date{3. März 2020}
%\institute[Uni Trier]{Universität Trier}
%\logo{\includegraphics[scale=.25]{unilogo.pdf}}

\begin{document}
	
\frame{\maketitle}
\frame{\frametitle{}
\tableofcontents
}

\section{Flashspeicher}
\frame{\frametitle{Besonderheiten des Flashspeichers}
\begin{itemize}
    \item Spezielle Struktur der Speicherzellen
    
    \item[ ]
    
    \pause
    \item Adressen Mapping
    \begin{itemize}
    \item Zuweisung von logischen zu physischen Adressen
    \item Häufige Verwendung wegen Charakteristika von Flash-Speichern
    \end{itemize}
    
    \item Garbage Collection
    \begin{itemize}
    \item Alte Daten/Seiten werden als ungültig markiert (Allokierter Speicherplatz)
    \item Hoher Aufwand durch kopieren von Blöcken
    \end{itemize}
    
    \item Wear Leveling
    \begin{itemize}
    \item Begrenzte Haltbarkeit von Flash-Zellen (Löschen, Schreiben)
    \item Gleichmäßige Abnutzung der Zellen
    \end{itemize}
    
    \item[ ]
    \pause
    \item Verwendung eines Flash Translation Layer (FTL) im Controller
\end{itemize}
}

\section{btrfs}
{\usebackgroundtemplate{%
		\hspace*{9cm}\includegraphics[width=.3\paperwidth,height=.3\paperheight]{img/btrfs.jpg}} 
\frame{\frametitle{btrfs - Einleitung}
	\begin{outline}
		\1 Full name:	B-tree file system
		\visible<2->{\1 Introduced in:	Linux kernel 2.6.29, March 2009}
		\visible<3->{\1 Developed by: 	Facebook, Fujitsu, Fusion-IO, Intel, Linux Foundation, Netgear, Oracle Corporation, Red Hat, STRATO AG, SUSE, ...}
	\end{outline}
}

\frame{\frametitle{btrfs - Struktur}
	\begin{outline}
		\1 
	\end{outline}
}

\frame{\frametitle{btrfs - CoW}
	
}

\frame{\frametitle{btrfs - RAID}
	
}}

\section{F2FS}
\frame{\frametitle{F2FS}
\begin{itemize}
    \item Flash-Dateisystem von Samsung (veröffentlicht 2012)
    \item Entwickelt nur für Flash-Speicher (SD-Karte, SSDs, eMMC-Karten)
    \item Ziel: Optimierung der Performance und Lebenszeit von Flash-Speichern
    \item Entwickelt als Open-Source Projekt
    
    \item[ ] 
    \pause\item Verfolgt den Ansatz eine Log-structured File System (append-only logging)
    \item Arbeitet nicht auf ,,raw'' Flash-Zellen (Benötigt einen FTL)
    \item Viele Möglichkeiten zur Anpassung des Systems
    \item Verwendung von iNodes und Datenblöcken (Ähnlich zu UNIX)
    
    \item[ ] 
    \pause\item Verfügbar ab Linux Kernel 3.8  
    \item Verwendung in Huawei (2016), Galaxy Note 10, Google Nexus
\end{itemize}
}

\frame{\frametitle{F2FS - Flash-friendly on-disk Layout}
\begin{itemize}
    \item Einheiten: Blöcke, Segmente, Sektionen, Zonen
    \item Orientierung an FTL-Einheiten um Kosten zu Vermeiden
    \pause\item Metadaten: 
    \begin{itemize}
        \item Random Writes: Vorhalten in Arbeitsspeicher (Bei Checkpoints schreiben)
    \end{itemize}
    \pause\item Haupt-Speicherbereich:
    \begin{itemize}
        \item Aufgeteilt in Standardmäßig 4KB Blocks (Jeder Block ist Node- oder Data-Block)
        \item Node- und Data-Blocks liegen in verschiedenen Segmenten
    \end{itemize}
\end{itemize}
\visible<2->{
\begin{figure}[h]
	\centering
	\includegraphics[width=\textwidth]{Figure_F2FS_ODL.png}
	\caption{on-disk Layout F2FS}
\end{figure}
}
}

\frame{\frametitle{F2FS - Besonderheiten I}
\begin{itemize}
    \item Multi-Head Logging
    \begin{itemize}
        \item Mehrere aktive Logsegmente parallel (Standard 6) 
        \item Parallele Verwendung durch Architektur möglich (multi-Streaming Interface)
        \item Unterscheidung der Daten in hot/warm/cold Schema (Update Frequenz)
    \end{itemize}
    \item[ ] 
    \pause\item Kosten-Effiziente Index Struktur
    \begin{itemize}
        \item Verwendung einer neuartigen Indexes: note adress table (NAT)
        \item Zur Vermeidung des ,,wandering tree'' Problems 
        \begin{itemize}
            \item Nur Update des direct Node Block und NAT
            \item Reduktion der Updates um Schreiboperationen zu sparen 
        \end{itemize}
    \end{itemize}
    
    
\end{itemize}
}

\frame{\frametitle{F2FS - Besonderheiten  II}
\begin{itemize}
    \item Adaptive logging
    \begin{itemize}
        \item Append-only Logging : Standardmäßig (random writes werden sequentiell)
        \item Threaded Logging : Verwendung bei hoher Auslastung (random writes)
    \end{itemize}
    \item[ ] 
    \pause
	\item Garbage Collection
    \begin{itemize}
        \item On-Demand: Wenn nicht genügend Speicherplatz verfügbar ist
        \begin{itemize}
            \item Greedy: Auswahl des Opfersegments mit wenigsten gültigen Blöcken
        \end{itemize}
        \item Background: Bei geringer Auslastung des Systems von Kernel ausgeführt
        \begin{itemize}
            \item Kosten-Effizient: Auswahl durch Segment-Alter und Anzahl gültiger Blöcke
        \end{itemize}
    \end{itemize}
\end{itemize}
}

\frame{\frametitle{F2FS - Bewertung}

\begin{itemize}
    \item Vorteile
    \begin{itemize}
        \item Optimierung der Zusammenarbeit von FTL und Dateisystem
        \item Vermeidung des Wandering Tree Problems
        \item Anpassung des Dateisystems an System-Status
        \item Hohe Anzahl an Parametern um Dateisystem anzupassen
    \end{itemize}
    
    \pause\item Nachteile
    \begin{itemize}
        \item Nur für Flash-Speicher (mit einem FTL)
        \item FTL Qualität wichtiges Kriterium 
        \item Initialer hoher belegter Speicherplatz durch Metadaten
        \item Hohe CPU-Belastung beim Schreiben von Dateien
    \end{itemize}
\end{itemize}
	
}

\section{benchmarks}
\frame{\frametitle{benchmarks}
	
}
\frame{\frametitle{benchmarks - Results}
		\begin{table}[h]
	\begin{tabular}{r|r|r|r|r|r}

	\end{tabular}
	\label{tab:t100benchmark}
	\caption{Benchmark}
\end{table}
}

\frame{\frametitle{Literatur}

\printbibliography[]
}





\end{document}
