\documentclass[10pt, xcolor=x11names]{beamer}
%\usecolortheme{seagull}
\useoutertheme{infolines}
\usefonttheme[onlymath]{serif}
\setbeamertemplate{headline}[default]
\setbeamertemplate{navigation symbols}{}
\mode<beamer>{\setbeamertemplate{blocks}[rounded][shadow=true]}
\setbeamercovered{transparent}
%\setbeamercolor{block body example}{fg=blue, bg=black!20}

\usepackage[utf8]{inputenc}
\usepackage[]{csquotes}
\usepackage{amsmath}
%\usepackage{tikz, wasysym}
%\usepackage{graphicx}
%\usetikzlibrary{automata,positioning}
%\usepackage{hyperref}
%\usepackage{amsfonts}
%\usepackage{csquotes}
%\usepackage{tikz}
%\usetikzlibrary{arrows}
%\usetikzlibrary{arrows.meta}
%\usetikzlibrary{positioning}
%\usepackage{wrapfig}
%\usepackage{pgfplots}
%\usepackage{outlines}
%\usepackage{mathtools}
%\usepackage{xcolor}
%\usepackage{amsfonts}
%\usepackage{amssymb}
%\usepackage{makeidx}
%\usepackage{graphicx}


\usepackage{hyperref}
\author{Sven Fiergolla \& Tobias Dahlem}
\title[]{btrfs \& f2fs vs. ext4}
\subtitle[short version]{}
\date{3. März 2020}
%\institute[Uni Trier]{Universität Trier}
%\logo{\includegraphics[scale=.25]{unilogo.pdf}}

\begin{document}
	
\frame{\maketitle}
\frame{\frametitle{}
\tableofcontents
}

\section{Introduction - Flash memory}
\frame{\frametitle{Introduction}

}

\section{btrfs}
\frame{\frametitle{btrfs}

}

\frame{\frametitle{btrfs - Struktur}
	
}

\frame{\frametitle{btrfs - CoW}
	
}

\frame{\frametitle{btrfs - RAID}
	
}

\section{f2fs}
\frame{\frametitle{f2fs}
	
}

\frame{\frametitle{f2fs - Struktur}
	
}

\frame{\frametitle{f2fs - Besonderheiten}
	
}

\frame{\frametitle{f2fs - RAID}
	
}

\frame{\frametitle{f2fs - Wandering-tree Problem}
	
}

\section{benchmarks}
\frame{\frametitle{benchmarks}
	
}
\frame{\frametitle{benchmarks - Results}
		\begin{table}[h]
	\begin{tabular}{r|r|r|r|r|r}

	\end{tabular}
	\label{tab:t100benchmark}
	\caption{Benchmark}
\end{table}
}







\end{document}
